\documentclass[letterpaper,12pt]{article}

\usepackage{amsmath,amsfonts,mathtools}
\usepackage{amsthm}
\usepackage{graphicx}
\usepackage{hyperref}
\usepackage{enumitem}
\usepackage{float}

% For displaying code
\usepackage{listings}
\usepackage{color}

\definecolor{dkgreen}{rgb}{0,0.6,0}
\definecolor{gray}{rgb}{0.5,0.5,0.5}
\definecolor{mauve}{rgb}{0.58,0,0.82}
\definecolor{orangered}{rgb}{1,0.27,0}

% Settings for displaying code
\lstset{%
  language=bash,
  aboveskip=3mm,
  belowskip=3mm,
  basicstyle={\small\ttfamily},
  commentstyle=\color{dkgreen},
  frame=single,
  numbers=none,
  numberstyle=\tiny\color{gray},
  stringstyle=\color{mauve},
  keywordstyle=\color{orangered},
  emphstyle=\color{blue},
  showstringspaces=false,
  otherkeywords={=, +, [, ], (, ), \{, \}, *},
  % bash commands from:
  %http://www.math.montana.edu/Rweb/Rhelp/00Index.html
  emph={addgroup,adduser,alias,
  ant,
  apropos,apt-get,aptitude,aspell,awk,
  basename,bash,bc,bg,borgbackup,break,builtin,bzip2,cal,case,cat,cd,cfdisk,chgrp,
  chkconfig,chmod,chown,chroot,cksum,clear,cmake,cmp,comm,command,continue,convert,
  cp,cron,crontab,csplit,curl,cut,date,dc,dd,ddrescue,declare,df,diff,diff3,
  dig,dir,dircolors,dirname,dirs,dmesg,dstat,dtrace,dtruss,du,echo,egrep,eject,enable,env,
  ethtool,eval,exec,exit,expand,expect,export,expr,false,fdformat,
  fdisk,ffmpeg,fg,fgrep,file,find,fmt,fold,for,format,free,fsck,ftp,function,
  fuser,gawk,getopts,
  git,gocryptfs,
  grep,groups,gzip,
  gunzip,
  ,hash,head,help,history,hostname,htop,hyperfine,
  id,if,ifconfig,ifdown,ifup,import,install,iotop,ip,
  java, java6, java_cur
  join,journalctl,jq,jupyter,notebook,kbfs,kernprof,kill,killall,less,
  let,ln,lnav,local,locate,logger,log,show,logname,logout,look,lpc,lpr,lprint,lprintd,
  lprintq,lprm,ls,lsof,lstat,make,man,mkdir,mkfifo,mkisofs,mknod,mmv,more,mosh,
  mount,mtools,mtr,mv,mypy,
  mysql,ncdu,nethog,
  netstat,nice,nl,nohup,notify-send,
  noweb,noweave,
  nslookup,op,
  open,passwd,paste,pathchk,pdflatex,perf,pgrep,ping,pkill,popd,pr,printcap,printenv,
  printf,ps,pushd,pwd,pyflakes,python, python3,quota,quotacheck,quotactl,ram,rclone,rcp,read,
  readarray,readonly,reboot,remsync,rename,renice,return,rev,rg,rm,rmdir,
  rsync,rustup,scp,screen,sdiff,sed,select,seq,set,sftp,sha1sum,shellcheck,shift,shopt,shutdown,
  sleep,slocate,sort,source,split,ss,ssh,sshfs,strace,su,sudo,sum,
  svn, svn2git,
  symlink,sync,systemctl,
  tail,tar,tcpdump,tee,test,time,times,tldr,tmux,top,touch,tr,traceroute,trap,tree,true,
  tsort,tty,type,ulimit,umask,umount,unalias,uname,unexpand,uniq,
  units,
  unrar,
  unset,unshar,until,useradd,usermod,users,uudecode,uuencode,vagrant,valgrind,
  vdir,version,vi,vmstat,watch,wc,wget,whereis,which,while,who,whoami,write,xargs, xdg,
  zcat,zsh},
}

% Display tildes nicely
\lstset{
    literate={~} {$\sim$}{1}
}

\newcommand*{\lstitem}[1]{
  \setbox0\hbox{\lstinline{#1}}
  \item[\usebox0]
}

% Personal definitions
\newcommand{\lra}{\ensuremath{\longrightarrow{}}}
\newcommand{\vect}[1]{\mathbf{#1}}
\renewcommand{\qedsymbol}{\rule{0.7em}{0.7em}}
\newcommand{\tabitem}{~~\llap{\textbullet}~~}

% Theorem commands
\newtheorem{lem}{Lemma}
\newtheorem{thm}{Theorem}
\newtheorem{defn}{Definition}

% Set spacing
\setenumerate{itemsep=1.5pt,parsep=1.5pt,topsep=0.5pt}
\setlist{itemsep=1.5pt,parsep=1.5pt,leftmargin=1pt}
\setitemize{itemsep=1.5pt,parsep=1.5pt,topsep=0.5pt}

% set 1" margins on 8.5" x 11" paper
% top left is measured from 1", 1"
\topmargin       0in
\oddsidemargin   0in
\evensidemargin  0in
\headheight      0in
\headsep         0in
\topskip         0in
\textheight      9in
\textwidth       6.5in

\begin{document}
\title{CS131 Notes}
\author{Sean Wu}
\date{\today}
\maketitle

\tableofcontents

\pagebreak

% set spacing
\setlength{\parindent}{0em}
\setlength{\parskip}{1em}

\section{Introduction}
\subsection{What is Computer Vision and why is it hard}

\begin{description}
 \item[Computer Vision]: extracting info from digital images OR developing algorithms to understand image content for other applications
\end{description}
\begin{itemize}
 \item Computer Vision is a hard interdisciplinary problem that is still unsolved
       \begin{itemize}
        \item Hard to convert data storing RGB values in many pixels to semantic info (ex. this blob of black pixels is a chair)
       \end{itemize}
 \item Vision (extracting meaningful info) is harder than 3D modelling
\end{itemize}

\subsection{Definition of Vision and Comparisons to Human Vision}
\begin{description}
 \item[sensing device]: captures details from a scene
 \item[interpreting device]: processes image from sensing device to extract meaning
\end{description}

\begin{itemize}
 \item Humans use eyes as sensing devices while computers use cameras
 \item For sensing devices, computer vision is actually better than human vision because cameras can see infrared, have longer range, and capture greater detail
 \item For interpreting devices, the human brain is way more advanced than computer systems
\end{itemize}

\subsection{Human Vision Strengths and Weaknesses}
\begin{itemize}
 \item Human vision evolved to quickly recognize danger for survival
 \item It is very fast \lra $\sim150$ ms to recognize an animal
 \item For speed, humans \textit{focus} only on "relevant" \textit{areas of interest}
 \item Thus, small signals/changes in the background can be difficult to detect and segment
 \item Humans also use \textit{context} to infer clues
       \begin{itemize}
        \item Used to determine next area of focus, when to expect certain objects in certain positions, and colour compensation in shadows
        \item However, context can be used to trick human vision
       \end{itemize}
 \item Context is very hard to include in computer vision
\end{itemize}

\subsection{Extracting info from images}

\begin{itemize}
 \item 2 types of info extracted in computer vision: \textbf{measurements} and \textbf{semantic info}
\end{itemize}

\subsubsection{Measurement in Vision}
\begin{itemize}
 \item Robots scan surroundings to make a map of its environment
 \item Stereo vision gives depth information (like 2 eyes) using triangulation
       \begin{itemize}
        \item Depth info represented as a depth map
       \end{itemize}
 \item With multiple viewpoints of an object, a 3D surface can be created (or even a 3D model)
\end{itemize}

\subsubsection{Obtaining Semantic Info from Vision}
\begin{itemize}
 \item Labelling objects (or scene)
 \item Recognizing people, actions, gestures, faces
\end{itemize}

\subsection{Applications of Computer Vision}
\begin{itemize}
 \item Video special effects
 \item 3D object modelling
 \item Scene recognition
 \item Face detection
       \begin{itemize}
        \item Note: face recognition is harder than face detection
       \end{itemize}
 \item Optical Character Recognition (OCR)
 \item Reverse image search
 \item Vision based interaction (ex. Microsoft Kinect)
 \item Augmented reality
 \item Virtual reality
\end{itemize}





\end{document}
